%%%%%%%%%%%%%%%%%%%%%%%%%%%%%%%%%%%%%%%%%
% Simple Sectioned Essay Template
% LaTeX Template
% i
% This template has been downloaded from:
% http://www.latextemplates.com
%
% Note:
% The \lipsum[#] commands throughout this template generate dummy text
% to fill the template out. These commands should all be removed when 
% writing essay content.
%
%%%%%%%%%%%%%%%%%%%%%%%%%%%%%%%%%%%%%%%%%

%----------------------------------------------------------------------------------------
%	PACKAGES AND OTHER DOCUMENT CONFIGURATIONS
%----------------------------------------------------------------------------------------

\documentclass[12pt]{article} % Default font size is 12pt, it can be changed here
\usepackage[english]{babel}
\usepackage[square,sort,comma,numbers]{natbib}
\usepackage[utf8]{inputenc}
\usepackage{listings}
\usepackage{color}
\usepackage{caption}
\usepackage[dvips]{graphicx}
\usepackage{geometry} % Required to change the page size to A4
%\geometry{a4paper} % Set the page size to be A4 as opposed to the default US Letter
\usepackage{framed}
\usepackage{graphicx} % Required for including pictures
\usepackage{amssymb}
\usepackage{url}

\usepackage{float} % Allows putting an [H] in \begin{figure} to specify the exact location of the figure
\usepackage{wrapfig} % Allows in-line images such as the example fish picture

\usepackage{fancyhdr}
\pagestyle{fancy}
\fancyhf{}
\fancyhead[RO]{{Assignment - 2}} 
\fancyhead[LO]{Biomedical Image Processing}
%\fancyhead[RO]{{\leftmark}} 
\fancyfoot[LE,RO]{{ \thepage }}

%\usepackage{lipsum} % Used for inserting dummy 'Lorem ipsum' text into the template
\definecolor{grey}{rgb}{0.9,0.9,0.9}

\linespread{1.2} % Line spacing

%\setlength\parindent{0pt} % Uncomment to remove all indentation from paragraphs

\graphicspath{{./Pictures/}} % Specifies the directory where pictures are stored
\newcounter{ejercicioNo}
\begin{document}

%----------------------------------------------------------------------------------------
%	TITLE PAGE
%----------------------------------------------------------------------------------------

\begin{titlepage}

\newcommand{\HRule}{\rule{\linewidth}{0.5mm}} % Defines a new command for the horizontal lines, change thickness here

\center % Center everything on the page

\textsc{\LARGE Universidad Aut\'{o}noma de Madrid}\\[1.5cm] % Name of your university/college
%\textsc{\Large Proyecto de Sistemas Informaticos}\\[0.5cm] % Major heading such as course name
%\textsc{\large Departamento de Informatica}\\[0.5cm] % Minor heading such as course title
\textsc{\Large Computer Science Department}\\[0.5cm] % Minor heading such as course title

\HRule \\[0.4cm]
{ \huge \bfseries Biomedical Image Processing\\[0.5cm] Assignment - 2}\\[0.4cm] % Title of your document
\HRule \\[1.5cm]

%\begin{minipage}{0.4\textwidth}
%\begin{flushleft}
% \large
%\emph{Author:}\\
%Roberto  \textsc{Marabini Ruiz} % Your name
%\end{flushleft}
%\end{minipage}

%\begin{minipage}{0.4\textwidth}
%\begin{flushright} \large
%\emph{Supervisor:} \\
%Dr. James \textsc{Smith} % Supervisor's Name
%\end{flushright}
%\end{minipage}\\[4cm]

%{\large \today}\\[3cm] % Date, change the \today to a set date if you want to be precise

%\includegraphics{Logo}\\[1cm] % Include a department/university logo - this will require the graphicx package

\vfill % Fill the rest of the page with whitespace
%\begin{minipage}{0.4\textwidth}
\begin{flushright}
 \large
%\emph{Author:}\\
Roberto  \textsc{Marabini Ruiz} % Your name
\end{flushright}
%\end{minipage}

\end{titlepage}

%----------------------------------------------------------------------------------------
%	TABLE OF CONTENTS
%----------------------------------------------------------------------------------------

\tableofcontents % Include a table of contents

\newpage % Begins the essay on a new page instead of on the same page as the table of contents 

%----------------------------------------------------------------------------------------
%	OBJETIVOS
%----------------------------------------------------------------------------------------

\section {Goal}

To better understand the contrast transfer function in electron microscopy.


%------------------------------------------------
\section{Introduction} % Sub-section

The contrast transfer function describes how an object examined in a transmission electron microscope is imaged, essentially providing a description of distortions due to imperfect image formation by the microscope. By considering the recorded image as a CTF-degraded true object, describing the CTF allows the true object to be reverse-engineered. 

 In the following we introduce a first order approximation used to describe the CTF.


\begin{eqnarray*}
    CTF(\mathbf{R})& = &w \cos(\gamma(\mathbf{R})) - \sqrt{1-w^2}\sin(\gamma(\mathbf{R}))\\
   \gamma (\mathbf{R})& = & 180\lambda(-\Delta Z \|\mathbf{R}\| ^2 + \frac{Cs10^6\|\mathbf{R}\|^4\lambda^2}{2})\ in\ degrees
\end{eqnarray*}
where $\mathbf{R}$ is the spatial frequency, $\Delta Z$ denotes defocus in nm.,  $w$ is the percentage of amplitude contrast and $Cs$ is the spherical aberration in mm. Finally, the factor $10^6$ converts $Cs$ from mm to nm. and $\gamma (\mathbf{R})$ is termed in the specialized literature as wave aberration function.\\

\begin{minipage}{\linewidth}
\begin{framed}
\addtocounter{ejercicioNo}{1}
Questions \arabic{ejercicioNo}:
\begin{itemize}
   \item Using matlab plot the CTF using the following two sets of values:
        \begin{itemize}
            \item Cs=0.6 mm. accelerating voltage 300keV ($\lambda$=1.97 pm),  $\Delta Z = 39.7$ nm.
            \item Cs=0.6 mm. accelerating voltage 300keV ($\lambda$=1.97 pm),  $\Delta Z = 90$ nm.
        \end{itemize}
   \item For which frequencies the contrast is inverted?
\end{itemize}
\end{framed}
\end{minipage}

\begin{minipage}{\linewidth}
\begin{framed}
\addtocounter{ejercicioNo}{1}
Questions \arabic{ejercicioNo}:
\begin{itemize}
   \item Simulate the aberrations introduced by the microscope by convolving the Einstein image
   with the above CTF.
   \item restore partially the image by multiplying (in Fourier space) it by the sign of the CTF
   \item restore partially the image by multiplying (in Fourier space) it by the $\frac{1}{CTF +\epsilon}$   where $\epsilon$ is a small number that avoid division by zero
\end{itemize}
\end{framed}
\end{minipage}

\begin{minipage}{\linewidth}
\begin{framed}
\addtocounter{ejercicioNo}{1}
Questions \arabic{ejercicioNo}:
\begin{itemize}
   \item Create a uniform noise image and convolve it with the CTF. Plot the magnitude 
   of the Fourier transform, the rings you see are the so-called Thon rings. (Fourier 
   transform the image, multiply it by the CTF, plot the
   Fourier transform magnitude)
   \item What happen if you increase $\Delta Z$? Does the distance between rings
   increase or decrease with a defocus increment?
\end{itemize}
\end{framed}
\end{minipage}


\end{document}
