\documentclass[a4paper,10pt]{article}
%\documentclass[a4paper,10pt]{scrartcl}
\usepackage{graphicx}


\usepackage[utf8]{inputenc}

\title{B-Factor Computation}
\author{Roberto Marabini}
\date{}

\pdfinfo{%
  /Title    ()
  /Author   ()
  /Creator  ()
  /Producer ()
  /Subject  ()
  /Keywords ()
}

\begin{document}
\maketitle

\section{Theoretical Background}

Identification and interpretation of high resolution structural features are hindered by the contrast loss caused by experimental and computational factors. This contrast loss is traditionally modeled by a Gaussian decay of structure factors with a temperature factor, or B-factor. Standard restoration procedures usually sharpen the experimental maps by applying a Gaussian function with an inverse ad hoc B-factor.

\section {Calculation}

see [1] for details

\section {Objetive}
Using Matlab compute the B-factor  as described in [1].

\section {Bibliography}
[1] Sharpening high resolution information in single particle electron cryomicroscopy. by fernandez et al. doi:10.1016/j.jsb.2008.05.010

\end{document}
