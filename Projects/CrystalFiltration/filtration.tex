\documentclass[a4paper,10pt]{article}
%\documentclass[a4paper,10pt]{scrartcl}
\usepackage{graphicx}


\usepackage[utf8]{inputenc}

\title{Averaging by quasioptical filtering}
\author{Roberto Marabini}
\date{}

\pdfinfo{%
  /Title    ()
  /Author   ()
  /Creator  ()
  /Producer ()
  /Subject  ()
  /Keywords ()
}

\begin{document}
\maketitle

\section{Theoretical Background}

Averaging of images of identical specimens (motive) for the purpose of noise elimination 
has a long history in many research fields. According to theory, averaging
over a correct set of images is equivalent to the result of Fourier filtration
of a regular 2D montage of that set (using appropriate delta-functions as a mask)
to pass only the Fourier components associated with the signal. (Aebi 1973)



\section {Calculation}

A noisy 2D crystals should be quasi optically filtered in order to restore 
a noiseless motive.

The method requires:
\begin{itemize}
 \item Fourier transform the original image
 \item Mask the Fourier transform 
 \item Integrate the density in the masked region
 \item Perform an inverse Fourier transform

\end{itemize}

\section {Objetive}
Using Matlab implemented a quasi optical filtration of a 2D crystal

\section {Bibliography}
R Henderson, JM Baldwin, KH Downing, J Lepault, F Zemlin. Structure of purple membrane from halobacterium halobium: recording, measurement and evaluation of electron micrographs at 3.5 Å resolution. Ultramicroscopy 19:147-178, 1986.

\end{document}
