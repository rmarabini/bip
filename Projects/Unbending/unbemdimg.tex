\documentclass[a4paper,10pt]{article}
%\documentclass[a4paper,10pt]{scrartcl}
\usepackage{graphicx}


\usepackage[utf8]{inputenc}

\title{Crystal Unbending}
\author{Roberto Marabini}
\date{}

\pdfinfo{%
  /Title    ()
  /Author   ()
  /Creator  ()
  /Producer ()
  /Subject  ()
  /Keywords ()
}

\begin{document}
\maketitle

\section{Theoretical Background}

Biological crystals are seldom perfect, they present distortions
and stretching, are limited in extent and disordered.
These imperfections degrade the Fourier Transform (FT) and reduce
high resolution information. The degradation of the FT
produces a broadening of the peaks at the reciprocal lattice
(i.e., they are no longer a delta function but a Gaussian)
plus an attenuation of the high frequency terms. As filtering
(in Fourier space) is equivalent to averaging all the unit
cells in the crystal (in real space), this degradation makes
the average unit cell blurred. In other words, crystal imperfections
translate into small misalignments of the different
unit cells and the final average then turns out blurred.
Lattice unbending was developed by Henderson et al.
(1986), to correct these distortions and recover high resolution
information. 

\section {Calculation}

2D crystals should be corrected, termed unbending, by
identifying the displacement of the unit cells compared with the
ideal lattice by cross-correlation Fourier analysis by using a
reference area from the center of the crystal itself.

The method requires:
\begin{itemize}
 \item Create a reference image (center of the crystal itself)
 \item Fourier transform both the crystal and the reference
 \item Croscorrelate both images
 \item Calcualte the the relative translations of the unit cells compared with a reference image. 
 \item Create (by interpolation) a perfect crystal from the original image
\end{itemize}

\section {Objetive}
Using Matlab implemented 2D crystal Unbending

\section {Bibliography}
Henderson, R., Baldwin, J.M., Downing, K.H., Lepault, J., Zemlin, F.,
1986. Structure of purple membrane from Halobacterium halobium:
recording, measurement and evaluation of electron micrographs at
3.5  \AA resolution. Ultramicroscopy 19, 147–178.

\end{document}
